\documentclass{beamer}

% \usepackage{beamerthemesplit} // Activate for custom appearance

\title{Project title: Molecular dynamics in the NVE ensemble}
\author{Authors, Master course}
\date{\today}

\begin{document}

\frame{\titlepage}

\section[Outline]{}
\frame{Talk Outline (for ca. 10-14 slides): \tableofcontents}

%\section{Introduction}
%\subsection{Introduction and Motivations}

\section{Introduction/Motivation (1 slide)}
\frame
{
  \frametitle{Introduction and Motivation }
\begin{itemize}
\item Introduce the project and its context (real experiments?)
\item As a general rule for the Methods/Results parts: 1 slide = 1 topical point
\item Total talk time 20 minutes (10+10 for two students shared). 10 minutes for discussion
\item The slides are only a support for your presentation, so avoid reading from them during your talk
\item Of course, you can also use powerpoint or other software
\end{itemize}
}

\section{Particular questions/goals (1 slide)}
\frame
{
  \frametitle{Questions/goals }
\begin{itemize}
\item Question/goal 1: For example, what is the density distribution of the particles? 
\item Question 2..
\item ...
\end{itemize}
}




\section{Theoretical background/Methods and implementation}
\subsection{Theoretical background/Methods (2 - 3 slides)}
\frame
{
  \frametitle{Theoretical background }

\begin{itemize}
\item Focus on methods new and special to your project
\end{itemize}  
  
Example: \\  Atoms are interacting through a Lennard-Jones potential given by:
\begin{equation}
V_{LJ}(r_{ij}) = 4\epsilon\left( \left( \frac{\sigma}{r_{ij}}\right)^{12} - \left( \frac{\sigma}{r_{ij}}\right)^{6} \right)
\label{eq:ULJ}
\end{equation}

\begin{itemize}
\item Initialize position/velocity at the desired temperature
\item Start loop on Time
	\begin{itemize}
	\item update position
	\item calculate force at $t+\Delta t$
	\item update the velocities
	\end{itemize}

\end{itemize}

  }
\subsection{Implementation and challenges (2 - 3 slides)}
\frame
{
  \frametitle{Implementation }
  
  \begin{itemize}
  \item structure of the code
  \item exemple of a function
  \item 2 slides
  \end{itemize}


  }

\frame
{
  \frametitle{Challenges}
  1 slide to explain the major challenges of the implementation


  }
  
  \section{Results (3 - 4 slides)}
\frame
{
  \frametitle{Results: The radial distribution function}
  
  \begin{figure}[htbp]
   \centering
   \includegraphics[width=75mm]{RDF/rdf.pdf} % requires the graphicx package
   %\caption{Radial distribution function as a function of the $r$. The horizontal line is a guideline for the reader.}
  \end{figure}
\begin{itemize}
\item location of the first solvation shell at 1.6$\sigma$
\end{itemize}

  }

\section{Discussion and Summary/Outlook (1-2 + 1 slides)}
\frame
{
  \frametitle{Discussion}
 1-2 slides for discussing your results 
  }
  
\frame 
{
  \frametitle{Summary}

\begin{itemize}
  
\item   Summarize your work in 1 slide 
\item   Acknowledgements 
\item   Practice - Practice and Practice more.
\end{itemize}

  }
\end{document}
