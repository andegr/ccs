\documentclass[11pt, oneside]{article}   	% use "amsart" instead of "article" for AMSLaTeX format
\usepackage{geometry}                		% See geometry.pdf to learn the layout options. There are lots.
\geometry{letterpaper}                   		% ... or a4paper or a5paper or ... 
%\geometry{landscape}                		% Activate for rotated page geometry
%\usepackage[parfill]{parskip}    		% Activate to begin paragraphs with an empty line rather than an indent
\usepackage{graphicx}				% Use pdf, png, jpg, or eps§ with pdflatex; use eps in DVI mode
								% TeX will automatically convert eps --> pdf in pdflatex		
\usepackage{amssymb}

%SetFonts

%SetFonts


\title{Final project\\ {\it Molecular dynamics in the NVE ensemble} \\ in the Master course "Computational Physics"}
\author{by: author names}
%\date{}							% Activate to display a given date or no date

\begin{document}
\maketitle
\begin{abstract}


The structure of a liquid can be directed quantified from the radial distribution function. In this work, we simulated using molecular dynamics the structure of a LJ fluid. We found that the first solvation shell is located at 1.6 $\sigma$.
\end{abstract}
\newpage
\section{Introduction and motivation}

In this section, we should have the following:
\begin{itemize}
\item Presentation of the topic, and its context to possible real world systems
\item The objective of the project
\end{itemize}

The introduction/motivation of the project has to be limited to 1 page.

\section{Theoretical background} \label{sec:TB}

This section summarizes the theoretical background needed to understand the implementation and the analysis. All the equations should be numbered, and used to add comments in the code. 

- Example text -\\
..the interaction between particle $i$ and $j$ is governed by a Lennard-Jones potential, which read as \cite{Frenkel}: 
\begin{equation}
V_{LJ}(r_{ij}) = 4\epsilon\left( \left( \frac{\sigma}{r_{ij}}\right)^{12} - \left( \frac{\sigma}{r_{ij}}\right)^{6} \right)
\label{eq:ULJ}
\end{equation}
where $\epsilon$ is the well depth of the potential, $\sigma$ is the sum of their van der Waals radii, and $r_{ij}$ is the inter-particle separation. Thus, the component 'x' of the force acting on particle 'i' is given by
\begin{equation}
f_{x,i} = -\frac{dV_{LJ}(r_{ij})}{dx_i} = ...
\label{eq:FLJ}
\end{equation}
-\\

Please limit this section to a maximum of 2 pages. {\it If space is a problem, focus more on the details of the methods which are new and special to your project.} Do not hesitate to make literature references to the script or research papers/books. 

\section{Simulation and Analysis}

\subsection{Implementation}

Here you can explain the structure of your code, and how you stored the data. Make sure that, all the equations presented in section~\ref{sec:TB} are well commented in the code. 

-Example text -\\
The force between particles given by Eq.~\ref{eq:FLJ} is implemented in the routine \textit{ForceLJ()}.
-\\

\subsection{Analysis}

-Example text -\\
In 3D, the radial distribution function (RDF) is calculated using the number of particles $dn$ that are found at distance $r$ within the shell volume between $r$ and $d+dr$ \cite{Frenkel}.
-\\

Limit this section to a maximum of 2 pages.

\section{Results}

-Example text -\\
As observed by the RDF (see Fig.~\ref{fig:rdf}). the system behaves as liquid. The first solvation shell, defined as the location of the first minimum, is located at 1.6$\sigma$.

\begin{figure}[htbp]
   \centering
   \includegraphics[width=75mm]{RDF/rdf.pdf} % requires the graphicx package
   \caption{Radial distribution function $g(r)$ as a function of the radial distance $r$ between a pair of atoms in units of the LJ size $\sigma$. The horizontal (red, dashed) line is a guideline for the reader.}
   \label{fig:rdf}
\end{figure}

\begin{itemize}
\item Answer to all the questions. 
\item Make sure, if you introduce a figure, that you are discussing it in the text, and it has a self-contained figure caption, describing everything one sees in the figure.  
\item There is no page limit for the results, but keep it concise, by assembling meaningful figures.
\end{itemize}
-
\section{Discussion}

The discussion should not be longer than 1 page. It should summarize the highlights of the analysis and results, and also attempt to make some context to real world systems. 

\renewcommand{\refname}{Bibliography}
\begin{thebibliography}{99}

  \bibitem{Frenkel}   DAAN; FRENKEL and BEREND; SMIT: Understanding Molecular Simulation: From Algorithms to Application. Academic Press, Second edition (2002)
\end{thebibliography}

\end{document}  